\documentclass[11pt]{article}
\usepackage{fullpage}
\usepackage{graphicx}
\usepackage{float}
\usepackage{alltt}
\usepackage{url}
\usepackage{amsmath}
\usepackage{amssymb}
\usepackage{xspace}

\renewcommand{\Pr}{\ensuremath{\mathbf{Pr}}\xspace}

\newcommand{\tuple}[1]{\ensuremath{\langle #1 \rangle}\xspace}

\newcommand{\Enc}{\ensuremath{\mathsf{Enc}}\xspace}
\newcommand{\Dec}{\ensuremath{\mathsf{Dec}}\xspace}

\newcommand{\mmod}{\ensuremath{\;\mathrm{mod}}\;}

\newcommand{\PT}{\ensuremath{\mathsf{PT}}\xspace}
\newcommand{\CT}{\ensuremath{\mathsf{CT}}\xspace}
\newcommand{\Key}{\ensuremath{\mathsf{Key}}\xspace}

\newcommand{\CC}{\ensuremath{\mathcal{C}}\xspace}
\newcommand{\KK}{\ensuremath{\mathcal{K}}\xspace}
\newcommand{\MM}{\ensuremath{\mathcal{M}}\xspace}

\newcommand{\E}{\ensuremath{\mathbb{E}}\xspace}
\newcommand{\D}{\ensuremath{\mathbb{D}}\xspace}
\newcommand{\K}{\ensuremath{\mathbb{K}}\xspace}

\begin{document}

\thispagestyle{empty}

\noindent \textbf{CS 6324: Information Security}
\begin{center}
{\LARGE Project 1 - Encrypted File System}
\end{center}

\section*{Design Explanation}

\begin{description}
 \item[Meta-data design]
Describe the precise meta-data structure you store and where.  For example, you should describe in which physical file and which location (e.g., in ``/abc.txt/0'', bytes from 0 to 127 stores the user\_name, bytes from 128 to ... stores ...).

 \item[User authentication]
Describe how you store password-related information and conduct user authentication.

 \item[Encryption design]
Describe how files are encrypted, how files are divided up, and how encryption is performed. Explain why your design ensures security even though the adversary can read each stored version on the disk.

 \item[File length hiding]
Describe how your design hides the file length to the degree that is feasible without increasing the number of physical files needed.

 \item[Message authentication]
Describe how you implement message authentication, and in particular, how this interacts with encryption.

 \item[Efficiency]
Analyze the storage and speed efficiency of your design. Describe a design that offers maximum storage efficiency. Describe a design that offers maximum speed efficiency. Explain why you chose your particular design.
\end{description}


\section*{Pseudo Code}

Provide pseudo-code for the functions \textbf{create, length, read, write, check\_integrity, and cut}. From your description, any crypto detail should be clear.  Basically, it should be possible to check whether your implementation is correct without referring to your source code.  You may want to describe how password is checked separately since it is used by several of the functions.


\section*{Design Variations}

\begin{enumerate}

 \item Suppose that the only write operation that could occur is to append at the end of the file. How would you change your design to achieve the best efficiency (storage and speed) without affecting security?

 \item Suppose that we are concerned only about adversaries who steal the disks. That is, the adversary can read only one version of the same file. How would you change your design to achieve the best efficiency?

 \item Can you use the CBC mode in the EFS? If yes, describe how your design would change and analyze the efficiency of the resulting design. If no, describe why.

 \item Can you use the ECB mode in the EFS? If yes, describe how your design would change and analyze the efficiency of the resulting design. If no, describe why.

\end{enumerate}

\end{document}
